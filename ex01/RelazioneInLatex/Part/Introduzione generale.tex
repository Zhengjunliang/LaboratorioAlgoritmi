\section{Introduzione generale}

\subsection{Scelta degli algoritmi}
In questa relazione, esploreremo due importanti algoritmi di string matching: 
\begin{itemize}
    \item L'algoritmo di ricerca ingenua (string matching ingenuo)
    \item L'algoritmo di Knuth-Morris-Pratt (KMP)
\end{itemize}
La scelta di analizzare questi due algoritmi è basata sulla loro importanza nella risoluzione di problemi legati al confronto di stringhe. Entrambi offrono approcci diversi per affrontare il problema del rilevamento di sottostringhe all'interno di una stringa più grande.

\subsection{Obiettivi della relazione}
Lo scopo di questa relazione è fornire una panoramica dettagliata di come funzionano l'algoritmo di ricerca ingenua e l'algoritmo KMP. Esamineremo la loro spiegazione teorica, la documentazione del codice, gli esperimenti condotti per valutarne le prestazioni e l'analisi dei risultati sperimentali. 

\subsection{Struttura della relazione}
La relazione è suddivisa in quattro parti fondamentali:

\begin{itemize}
    \item \textbf{Spiegazione teorica dell'algoritmo}: in questa sezione, forniremo una descrizione teorica dell'algoritmo, compresi i principi di funzionamento e gli aspetti teorici.
    \item \textbf{Documentazione del codice}: qui esamineremo come l'algoritmo è stato implementato nel codice, discutendo le scelte di progettazione e l'approccio adottato.
    \item \textbf{Descrizione degli esperimenti condotti}: presenteremo gli esperimenti che abbiamo condotto per valutare le prestazioni degli algoritmi, inclusi i dettagli delle configurazioni e delle misurazioni.
    \item \textbf{Analisi dei risultati sperimentali}: alla luce dei dati raccolti dagli esperimenti, discuteremo le prestazioni degli algoritmi, ne valuteremo l'efficienza e trarranno conclusioni sulla loro efficacia in scenari di string matching reali.
\end{itemize}

\subsection{Ambiente di sviluppo}
Il codice per la realizzazione degli esperimenti è stato scritto in Python e l'IDE utilizzato è \textbf{PyCharm 2023.2.1}. Questa relazione è stata redatta con l'editor online \textbf{Overleaf}.
